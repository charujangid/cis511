\documentclass[11pt]{article}

\usepackage{tikz}

\begin{document}
\begin{enumerate}
\item
  \begin{enumerate}
  \item Base case:\\
    $i=0: Q^0_r$\\
    A path of length zero consists of the start state.\\
    Inductive Hypotheses:\\
    Assume $Q^i_r$ is the set of all states reachable from $q_0$ with paths of length i.\\
    Inductive Step:\\
    Prove $Q^{i+1}_r$ is the set of all states reachable from $q_0$ with paths of length i + 1.\\
    Using the definition, $Q^{i+1}_r = \{ q \in Q \rule \exists p \in Q^i_r, \exists a \in \Sigma, q = \delta\(p,a\) \}$ and the inductive hypothesis, every state in $Q^{i+1}_r$ must be obtained by consuming a letter in $\Sigma$. And since we know that every state in $Q^i_r$ is reachable in i steps, every state $q \in Q^{i+1}_r$ is reachable from the start state with a path of length i + 1.
  \item draw later
  \item draw later
  \item Proof by contradiction: assume there is no smallest integer such that $Q^{i_0+1}_r = Q^{i_0}_r = Q_r$.\\
    Because $\rule Q^{i_0+1}_r \rule >= \rule Q^{i_0}_r \rule$, and we assumed that $Q^{i_0+1}_r \noteq Q^{i_0}_r$, $\rule Q^{i_0+1}_r \rule > \rule Q^{i_0}_r \rule$. Therefore, as $i$ increases infinitely, so does the cardinality of $Q^r$, which contradicts the fact that DFAs must have a finite number of states. QED. FIX FIX FIX FIX FIX\\\\

    Why is $D_r$ a DFA?\\
    Because it consists of all reachable states from the original DFA. In other words, it's the original DFA minus all unreachable states. FIX FIX FIX\\\\

    Proof by contradiction:\\
    Assume that $L(D_r) \noteq L(D)$.\\
    Direction 1: $w \in L(D)$\\
    Direction 2: $w \in L(D_r)$\\
  \end{enumerate}
\item Bezout + diophantine + quotient / remainder
\item 1. Induction on |u| + |v| 2. some property of addition
\item 
\end{enumerate}
\end{document}
